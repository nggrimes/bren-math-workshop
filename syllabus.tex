% Options for packages loaded elsewhere
\PassOptionsToPackage{unicode}{hyperref}
\PassOptionsToPackage{hyphens}{url}
\PassOptionsToPackage{dvipsnames,svgnames,x11names}{xcolor}
%
\documentclass[
  letterpaper,
  DIV=11,
  numbers=noendperiod]{scrartcl}

\usepackage{amsmath,amssymb}
\usepackage{iftex}
\ifPDFTeX
  \usepackage[T1]{fontenc}
  \usepackage[utf8]{inputenc}
  \usepackage{textcomp} % provide euro and other symbols
\else % if luatex or xetex
  \usepackage{unicode-math}
  \defaultfontfeatures{Scale=MatchLowercase}
  \defaultfontfeatures[\rmfamily]{Ligatures=TeX,Scale=1}
\fi
\usepackage{lmodern}
\ifPDFTeX\else  
    % xetex/luatex font selection
\fi
% Use upquote if available, for straight quotes in verbatim environments
\IfFileExists{upquote.sty}{\usepackage{upquote}}{}
\IfFileExists{microtype.sty}{% use microtype if available
  \usepackage[]{microtype}
  \UseMicrotypeSet[protrusion]{basicmath} % disable protrusion for tt fonts
}{}
\makeatletter
\@ifundefined{KOMAClassName}{% if non-KOMA class
  \IfFileExists{parskip.sty}{%
    \usepackage{parskip}
  }{% else
    \setlength{\parindent}{0pt}
    \setlength{\parskip}{6pt plus 2pt minus 1pt}}
}{% if KOMA class
  \KOMAoptions{parskip=half}}
\makeatother
\usepackage{xcolor}
\setlength{\emergencystretch}{3em} % prevent overfull lines
\setcounter{secnumdepth}{-\maxdimen} % remove section numbering
% Make \paragraph and \subparagraph free-standing
\ifx\paragraph\undefined\else
  \let\oldparagraph\paragraph
  \renewcommand{\paragraph}[1]{\oldparagraph{#1}\mbox{}}
\fi
\ifx\subparagraph\undefined\else
  \let\oldsubparagraph\subparagraph
  \renewcommand{\subparagraph}[1]{\oldsubparagraph{#1}\mbox{}}
\fi


\providecommand{\tightlist}{%
  \setlength{\itemsep}{0pt}\setlength{\parskip}{0pt}}\usepackage{longtable,booktabs,array}
\usepackage{calc} % for calculating minipage widths
% Correct order of tables after \paragraph or \subparagraph
\usepackage{etoolbox}
\makeatletter
\patchcmd\longtable{\par}{\if@noskipsec\mbox{}\fi\par}{}{}
\makeatother
% Allow footnotes in longtable head/foot
\IfFileExists{footnotehyper.sty}{\usepackage{footnotehyper}}{\usepackage{footnote}}
\makesavenoteenv{longtable}
\usepackage{graphicx}
\makeatletter
\def\maxwidth{\ifdim\Gin@nat@width>\linewidth\linewidth\else\Gin@nat@width\fi}
\def\maxheight{\ifdim\Gin@nat@height>\textheight\textheight\else\Gin@nat@height\fi}
\makeatother
% Scale images if necessary, so that they will not overflow the page
% margins by default, and it is still possible to overwrite the defaults
% using explicit options in \includegraphics[width, height, ...]{}
\setkeys{Gin}{width=\maxwidth,height=\maxheight,keepaspectratio}
% Set default figure placement to htbp
\makeatletter
\def\fps@figure{htbp}
\makeatother

\addtokomafont{disposition}{\rmfamily}
\KOMAoption{captions}{tableheading}
\makeatletter
\@ifpackageloaded{caption}{}{\usepackage{caption}}
\AtBeginDocument{%
\ifdefined\contentsname
  \renewcommand*\contentsname{Table of contents}
\else
  \newcommand\contentsname{Table of contents}
\fi
\ifdefined\listfigurename
  \renewcommand*\listfigurename{List of Figures}
\else
  \newcommand\listfigurename{List of Figures}
\fi
\ifdefined\listtablename
  \renewcommand*\listtablename{List of Tables}
\else
  \newcommand\listtablename{List of Tables}
\fi
\ifdefined\figurename
  \renewcommand*\figurename{Figure}
\else
  \newcommand\figurename{Figure}
\fi
\ifdefined\tablename
  \renewcommand*\tablename{Table}
\else
  \newcommand\tablename{Table}
\fi
}
\@ifpackageloaded{float}{}{\usepackage{float}}
\floatstyle{ruled}
\@ifundefined{c@chapter}{\newfloat{codelisting}{h}{lop}}{\newfloat{codelisting}{h}{lop}[chapter]}
\floatname{codelisting}{Listing}
\newcommand*\listoflistings{\listof{codelisting}{List of Listings}}
\makeatother
\makeatletter
\makeatother
\makeatletter
\@ifpackageloaded{caption}{}{\usepackage{caption}}
\@ifpackageloaded{subcaption}{}{\usepackage{subcaption}}
\makeatother
\ifLuaTeX
  \usepackage{selnolig}  % disable illegal ligatures
\fi
\usepackage{bookmark}

\IfFileExists{xurl.sty}{\usepackage{xurl}}{} % add URL line breaks if available
\urlstyle{same} % disable monospaced font for URLs
\hypersetup{
  pdftitle={Calculus Workshop 2024 Syllabus},
  pdfauthor={Nathaniel Grimes},
  colorlinks=true,
  linkcolor={blue},
  filecolor={Maroon},
  citecolor={Blue},
  urlcolor={Blue},
  pdfcreator={LaTeX via pandoc}}

\title{Calculus Workshop 2024 Syllabus}
\usepackage{etoolbox}
\makeatletter
\providecommand{\subtitle}[1]{% add subtitle to \maketitle
  \apptocmd{\@title}{\par {\large #1 \par}}{}{}
}
\makeatother
\subtitle{Bren MESM Orientation}
\author{Nathaniel Grimes}
\date{}

\begin{document}
\maketitle

Email: nggrimes@ucsb.edu

Office: Bren Hall 3007

Location: Bren Hall 1414

\subsection{Course Objectives}\label{course-objectives}

Bren students come from all walks of life. Some incoming students might
be twenty year career professionals in everything but math. Others might
be right out of their math Bachelors degree armed with multivariate
calculus and real analysis. This workshop is meant to get everyone to a
sufficient level to excel in all the courses offered at Bren.
Collaboration is essential during your MESM program. This workshop will
build teams so that everyone can benefit from diverse backgrounds in
both math and life. The primary workshop objectives are as follows:

\begin{itemize}
\item
  Reinforce math skills used in Bren courses
\item
  Understand how math and calculus are used in Environmental Science
\item
  Help transition to graduate school life
\item
  Build collaborative environment
\end{itemize}

\subsection{Workshop design}\label{workshop-design}

New research has shown that active learning outside of traditional
lecture formats boosts student engagement, understanding, and comfort.

We'll be using a Team Based Learning (TBL) environment. Students will be
grouped into teams of 4-5.
\href{https://docs.google.com/forms/d/13qlU8WEgVMUxiuF9gezNgu5oAbgDX4zeN1QcjbZssfM/edit}{Please
take this survey} so I can balance teams across math comfort,
background, and when the last time you did a problem set. Teams will
work and collaborate together throughout the workshop.

\subsubsection{Student Responsibilities}\label{student-responsibilities}

Students are responsible for showing up adequately prepared for each
lecture by completing problem sets and watching pre-class videos posted
on course website.

Students actively support and encourage teammates to improve learning
for all. Students are encouraged to approach the instructor to help
resolve conflicts if they cannot be resolved internally.

Students must be active participants within their team to deepen
engagement.

\subsubsection{Class strucutre}\label{class-strucutre}

Here is a sample framework each class will generally follow.

\begin{itemize}
\item
  10 minutes team problem set debrief
\item
  10 minute class wide review
\item
  15-20 minute new instruction
\item
  10 minute Team assessment
\item
  5-10 minute break
\item
  20 minute instruction
\item
  10 minute Team assessment
\item
  15-20 minute Team environmental problem application
\end{itemize}

\subsubsection{Work outside class}\label{work-outside-class}

Typically TBL environments consist of outside learning modules that
students can walkthrough at their own pace. However, given the 8-5
schedule of orientation week and the far more important socializing that
occurs after hours, we'll be dropping those elements. Instead, short
problem sets no more than 8 questions will be given online. I will
reference some specific Khan Academy or Paul's notes resource to help
students prepare for the next day. But that will be the extent of
outside work.

\subsection{Course Materials}\label{course-materials}

Course materials will be accessible on the course website. Videos,
problem sets, answer keys, and slides will be hosted there. In addition,
any code generated for the course will be hosted on the workshop public
\href{https://github.com/nggrimes/bren-math-workshop}{github repo}.
Coding is a skill that will be taught at Bren through ESM 206, data
bootcamp, and ESM 244. Seeing more examples of codes always helps to
understand how it works.

\subsection{Schedule and Topic
Outline}\label{schedule-and-topic-outline}

Thursday September 19, 2024 \textbf{12:00 P.M. - 2:00 P.M.}:
Introduction, Algebra and pre-calculus review, e.g.~exponents and
logarithms

Friday September 20, 2024 \textbf{9:00 A.M. - 11:00 A.M.}: Graphing,
Limits, and Fundamentals of Calculus

Friday September 20, 2024 \textbf{12:00 P.M. - 2:00 P.M.}: Rules and
Applications of derivatives

Monday September 23, 2024 \textbf{9:00 A.M. - 11:00 A.M.}: Integration

Tuesday September 24, 2024 \textbf{12:30 P.M. - 2:30 P.M.}: Differential
Equations and Computational Calculus, i.e.~let a computer do the work

\subsection{Grading and participation}\label{grading-and-participation}

This workshop is not mandatory, though highly encouraged. The team
environment will deepen your own individual understanding even if you
have a mastery of calculus as you will help your teammates along the
way. Nothing will be graded, but each problem set will include immediate
feedback and answer keys.

\subsection{Office Hours}\label{office-hours}

I will be available following each class period to discuss anything,
including topics beyond math. If you would like extra time to meet
please email me to arrange a time.

\subsection{Extra Resources}\label{extra-resources}

It is impossible to cram 3 quarters of math into 5 days. Here is a brief
list of external resources if you want to brush up on more math topics.
This will be added to as resources are identified. If students find
additional resources that are useful to them, please feel free to share!

\href{https://www.khanacademy.org/}{Khan Academy} - Excellent collection
of videos with in depth explanations on topics we covered here and
beyond. The Calculus sections are particularly well thought out.

\href{https://www.wolfram.com/mathematica/}{Wolfram Mathematica} -
Online Equation solver. Can be cumbersome to use, but good to check
solutions if worried about algebra mistakes.

\href{https://tutorial.math.lamar.edu/}{Paul's Math Notes} Dr.~Paul
Dawkins at Lamar University has an awesome collection of all his lecture
notes for his college level classes available here.



\end{document}
